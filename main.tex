\documentclass[11pt]{IEEEtran}
\usepackage{graphicx}
\usepackage{amsmath}
\usepackage{cite}
\usepackage{booktabs}
\usepackage{hyperref}

\markboth{PV2023: Adding value (to) and preserving Scientific & Technical data, 2-4 May 2023 CERN}{Last Name \MakeLowercase{\textit{et al.}}: Title}

\title{ The Challenge of Digital Preservation at CERN }
\author{Antonio Vivace, Jean-Yves Le Meur\\
    \{antonio.vivace, jean-yves.le.meur\}@cern.ch}
\begin{document}
\maketitle

\begin{abstract}
This paper outlines the challenge of preserving the vast and diverse amount of digital content generated by the CERN community.
To begin, we define the initial scope of preservation, giving an overview of the digital repositories selected for the development of an OAIS compliant archive at CERN.

The OAIS reference model will then be briefly introduced, explaining what are the fundamental components of a digital preservation system and how they are being  implemented.

TODO: Rescope OAIS here

Next, we introduce the CERN Digital Preservation platform, which is designed to orchestrate these components and enable existing information systems to adopt a preservation strategy. We highlight how existing tools and digital preservation solutions, such as Archivematica, are being integrated into this platform.

\end{abstract}

\section{Preservation scope}

Scientific and institutional data produced at CERN is disseminated on a number of different repositories and internal or external content management systems.

Below are listed the information systems considered during this initial phase.

\subsection{Institutional Repositories}

\textbf{Zenodo}, a general general-purpose open repository hosting research papers, data sets, research software. Targeted at a wider world audience. It is based on the InvenioRDM \cite{InvenioRDMinveniosoftwareorg-2023-03-16} software.

\textbf{CERN Document Server}, a repository dedicated to documenting articles, reports, and multimedia content in High Energy Physics (HEP). It includes articles and preprints, books and proceedings, presentation and talks. it is powered by an old version of the Invenio software.

\textbf{Indico}, a powerful event (meetings, conferences, lectures) management system.

\textbf{CERN Open Data portal}, the access point to a growing range of data produced through the research performed at CERN. It disseminates the preserved output from various research activities and includes accompanying software and documentation needed to understand and analyse the data.

\subsection{Other systems}


\textbf{GitLab} is a web-based Git repository and collaborative software development platform that code hosting, project management capabilities, wikis, continuos deployment and integration pipelines. A large part of the software developed at CERN experiments and IT projects is hosted on a CERN instance of GitLab.

\textbf{CodiMD} provides a way to quickly edit and share notes and small documents in Markdown. It is often used to prepare internal drafts or informal documentation.


\section{Methodology}

The fundamental standard for digital preservation is the OAIS Reference Model\cite{OAIS2002} (ISO 14721). It provides requirements and standards for an archive or repository to provide long-term, preservation of digital information.



A preliminary inspection was conducted on the mentioned sources, showing that they are lacking the strategies and the capabilities to digitally preserve their assets (namely, the capacity to prepare Archival Information Packages and disseminate them).





\subsection{Environment}

No enforced policy. 3 options proposed. \cite{strategy}


\subsubsection{Preservation is performed by the IR} the Information Repository (IR) must follow the OAIS model to preserve its content and perform preservation actions for data types within its preservation scope. The IR does not require a dedicated central service as it handles preservation actions and registry internally. A central entity will verify that CERN policy is implemented within the IR and may harvest the IR's preservation registry to feed into a central registry.

\subsubsection{Preservation is shared between the IR and the Digital Preservation Service}

The Information Repository (IR) is responsible for creating Archival Information Packages (AIP) in formats aligned with CERN specifications and for performing migrations when required. These AIPs are deposited in a central digital preservation Registry maintained by the Digital Preservation (DP) service. DP provides versioning and transfer of AIPs to the CERN Tape Archive or remote long-term storage for bit preservation. The bags produced by the IR are never modified by the Preservation service but need to be validated against a central specification before being archived.

\subsubsection{Preservation is delegated to the Digital Preservation service}

The Digital Preservation service supports an “AIP Factory”. It provides IRs with a protocol to push or pull the original content to be preserved according to a Submission Information Package (SIP) specification . Preservation processes are performed according to current standards using existing specialized software (such as Archivematica). The final AIPs are stored on tape and can be retrieved at any time by the submitting IR. Optional services, such as the generation of dissemination packages (DIP) with, for example, OCR files, proxy formats, etc can be added on request.


\subsection{Workflow}

TODO: Describe the option where we harvest the SIP.

\subsection{User workflow}

We also implemented a \textit{self service area}, where CERN users can autonomously login using their credentials (through the CERN SSO) and search for their records across a number of different supported sources, such as Indico and the CERN Document Server. This allows the user to:

\begin{itemize}
    \item select what they may want to archive 
    \item add some additional metadata and curation
    \item organize what they are archiving applying "tags"
    \item let the platform fetch the resources they requested 
    \item download archival packages for their personal usage
    \item submit assets for long term preservation
\end{itemize}

This aims to cover the cases of users leaving CERN, changing projects or experiments, changing their contracts or simply wanting to do a periodical review of what they produced in the last period (e.g. during their MERIT process).

\subsection{Implementation}

CERN SIP \cite{CERNDigitalMemorySIPSpecGitLab-2021-09-15}, BagIt Create \cite{CERNDigitalMemorybagitcreateGitLab-2023-02-17}

\section{Results}


\subsection{ILCDoc}

\subsection{old dh head}

\subsection{easy deployment with Kubernetes/Helm charts}

\subsection{people retiring}



\section{Conclusion}

cta/how much/when/decisions

moving api surface

selecting content?

policies at the level of the source information systems

archivematica, support for office documents

access to archives

\bibliographystyle{IEEEtran}
\bibliography{references}

\end{document}
